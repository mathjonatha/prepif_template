\documentclass[10pt]{article}
\usepackage[T1]{fontenc}
\usepackage[utf8]{inputenc}
\usepackage[brazil]{babel}
%\usepackage[math]{anttor} % fonte um pouco mais estilizada
\usepackage{import}
%\usepackage{parskip}
%=========================Packages==================================%
\usepackage{lscape,booktabs,latexsym,multicol,lmodern, natbib,graphicx,tikz,tkz-euclide,enumitem,fancyhdr, lipsum,siunitx, setspace,float}

\usepackage{amsmath,amsfonts,amssymb,amsthm}
\everymath{\displaystyle}

% configurações das questões, bem como: pontuação e estrutura.

\usepackage{tasks} % cria lista curta
\usepackage{exsheets} % cria questoes
\SetupExSheets[points]{name=ponto/s,number-format=\color{blue}} % define as configurações de pontuação das questões, e a cor da pontuação.

\DeclareInstance{exsheets-heading}{fancy-wp}{default}{
toc-reversed = true ,
indent-first = true ,
vscale = 2 ,
pre-code = \rule{\linewidth}{1pt} ,
post-code = \rule{\linewidth}{1pt} ,
title-format = \large\scshape\color{rgb:red,0.65;green,0.04;blue,0.07} ,
number-format = \large\bfseries\color{rgb:red,0.02;green,0.04;blue,0.48} ,
points-format = \itshape ,
points-pre-code = ( ,
points-post-code = ) ,
join =
{
number[r,B]title[l,B](.333em,0pt) ;
number[r,B]points[l,B](.333em,0pt)
} ,
attach = { main[hc,vc]number[hc,vc](0pt,0pt) }
}

%\SetupExSheets{headings=fancy-wp} % estilo diferente para o topo do enunciado com o nome " Exercício




\graphicspath{{imgs/}} %informa a pasta em que as imagens estão
%\usepackage{showframe} %Mostra linhas de marcação e margens
%FONTES
%\usepackage{fontspec}
%\usepackage{avant}
%\usepackage{mathptmx}
\usepackage{times}
%\usepackage[classicRfeIm]{kpfonts}
%\usepackage{kurier}


\usepackage{hyperref}% add hypertext capabilities
%\usepackage{txfonts}
%\usepackage{mathrsfs}

% CORES
\usepackage{xcolor}
\definecolor{corprimaria}{RGB}{10,48,123} 
\definecolor{corsecundaria}{RGB}{116,23,255}
\definecolor{corlinha}{RGB}{47,158,65}

\definecolor{corexercicio}{RGB}{38,38,38}
\definecolor{cordefinicao}{RGB}{47,158,65}
% \definecolor{corexemplo}{RGB}{116,23,255}
\definecolor{corexemplo}{RGB}{38,38,38}

% CORES DO PREPIF
\definecolor{cor1}{RGB}{239,239,239}
\definecolor{cor2}{RGB}{205,25,30}
\definecolor{cor3}{RGB}{47,158,65}
\definecolor{cor4}{RGB}{38,38,38}
\definecolor{cor5}{RGB}{28,28,28}

% ESTILOS
\newtheoremstyle{geral}% Nome do estilo do teorema
{0pt}% Espaço acima
{0pt}% Espaço abaixo
{\normalfont}% Fonte do corpo
{}% Indent amount
{\small\bf\sffamily\color{cor4}}% Theorem head font
{:}% Punctuation after theorem head
{0.25em}% Space after theorem head
{} % Optional theorem note

\newcounter{dummy}
\newcounter{Exer}
\newcounter{exer}
\newcounter{exem}
\theoremstyle{geral}
\newtheorem{ExercicioT}[Exer]{Exercício resolvido}
% \newtheorem{exercicioT}[exer]{ }
\newtheorem{definicaoT}[dummy]{Conceito}
\newtheorem{exemploT}[exem]{Exemplo}
\newtheorem{obsT}{Observação}

\RequirePackage[framemethod=default]{mdframed} % Required for creating the theorem, definition, exercise and corollary boxes

% Caixa de exercícios
\newmdenv[skipabove=7pt,
skipbelow=7pt,
rightline=false,
leftline=true,
topline=false,
bottomline=false,
backgroundcolor=corexercicio!10,
linecolor=corexercicio,
innerleftmargin=5pt,
innerrightmargin=5pt,
innertopmargin=5pt,
innerbottommargin=5pt,
leftmargin=0cm,
rightmargin=0cm,
linewidth=4pt]{caixaEx}	

% Caixa de definição
\newmdenv[skipabove=7pt,
skipbelow=7pt,
rightline=false,
leftline=true,
topline=false,
bottomline=false,
backgroundcolor=cordefinicao!10,
linecolor=cordefinicao,
innerleftmargin=5pt,
innerrightmargin=5pt,
innertopmargin=5pt,
innerbottommargin=5pt,
leftmargin=0cm,
rightmargin=0cm,
linewidth=4pt]{caixaDE}	

% Caixa de exemplos
\newmdenv[skipabove=7pt,
skipbelow=7pt,
rightline=false,
leftline=true,
topline=false,
bottomline=false,
backgroundcolor=corexemplo!10,
linecolor=corexemplo,
innerleftmargin=5pt,
innerrightmargin=5pt,
innertopmargin=5pt,
innerbottommargin=5pt,
leftmargin=0cm,
rightmargin=0cm,
linewidth=4pt]{caixaExem}	

% Caixa de observações	  
\newmdenv[skipabove=7pt,
skipbelow=7pt,
rightline=false,
leftline=true,
topline=false,
bottomline=false,
backgroundcolor=cor2!10,
linecolor=cor2,
innerleftmargin=5pt,
innerrightmargin=5pt,
innertopmargin=5pt,
innerbottommargin=5pt,
leftmargin=0cm,
rightmargin=0cm,
linewidth=4pt]{caixaObs}


\newenvironment{Exercicio}{\begin{caixaEx}\begin{ExercicioT}}{\end{ExercicioT}\end{caixaEx}}
% \newenvironment{exercicio}{\begin{caixaEx}\begin{exercicioT}}{\end{exercicioT}\end{caixaEx}}
\newenvironment{definicao}{\begin{caixaDE}\begin{definicaoT}}{\end{definicaoT}\end{caixaDE}}
\newenvironment{exemplo}{\begin{caixaExem} \begin{exemploT}}{\end{exemploT}\end{caixaExem}}

\newenvironment{obs}{\begin{caixaObs}\begin{obsT}}{\end{obsT}\end{caixaObs}}

\newenvironment{exercicio}[2][{\color{corexercicio}Exercício}]{\begin{trivlist}

\item[\hskip \labelsep {\bfseries #1}\hskip \labelsep {\bfseries #2.}]}{\end{trivlist}}


%===================================================
% MARGINS
%\usepackage[top=8mm, bottom=20mm, left=8mm, right=8mm]{geometry}
\usepackage{geometry}
\geometry{
	paper=a4paper, 
	top=2.5cm, 
	bottom=2.5cm, 
	left=1.5cm, 
	right=2cm,
	headheight=14pt, % Header height
	footskip=1.4cm, % Espaço da margem inferior à linha de base do rodapé
	headsep=10pt, % Espaço da margem superior até a linha de base do cabeçalho
	%showframe, % Uncomment to show how the type block is set on the page
}


% NOVOS COMANDOS
\newcommand{\atv}{Lista de Exercícios 00}
\newcommand{\preceptor}{Monitor: Matheus Jonatha}




% CABEÇALHO E RODAPÉ
\pagestyle{fancy}
%\lfoot{\notaesquerda}
\cfoot{}
\rfoot{{\color{white}\thepage}}
%\lhead{HELLO}
%\chead{HELLO}
%\rhead{\textbf{The performance of new graduates}}
\renewcommand{\headrulewidth}{0pt} %linha horizontal no topo da pagina
%\renewcommand{\footrulewidth}{0.4pt} %linha horizontal no pé da pagina

%\setlength\parindent{0pt}
%\setlength\parskip{1.5ex}
%\setlength\parsep{1.5\parskip}
%\thispagestyle{empty}%\bigskip %Rodapé na primeira pagina


%para nao ficar o retangulo em volta dos links, apenas muda a cor dos caracteres
\hypersetup{ colorlinks,
linkcolor=blue,
filecolor=blue,
urlcolor=blue,
citecolor=blue }



% QUESTÕES
% configurações das questões, bem como: pontuação e estrutura.

\usepackage{tasks} % cria lista curta
\usepackage{exsheets} % cria questoes
\SetupExSheets[points]{ name=ponto/s,number-format=\color{blue}} % define as configurações de pontuação das questões, e a cor da pontuação.

\DeclareInstance{exsheets-heading}{fancy-wp}{default}{
toc-reversed = true ,
indent-first = true ,
vscale = 2 ,
pre-code = \rule{\linewidth}{1pt} ,
post-code = \rule{\linewidth}{1pt} ,
title-format = \large\scshape\color{rgb:red,0.65;green,0.04;blue,0.07} ,
number-format = \large\bfseries\color{rgb:red,0.02;green,0.04;blue,0.48} ,
points-format = \itshape ,
points-pre-code = ( ,
points-post-code = ) ,
join =
{
number[r,B]title[l,B](.333em,0pt) ;
number[r,B]points[l,B](.333em,0pt)
} ,
attach = { main[hc,vc]number[hc,vc](0pt,0pt) }
}

%\SetupExSheets{headings=fancy-wp} % estilo diferente para o topo do enunciado com o nome " Exercício
\usepackage{capt-of}%%To get the caption

\hypersetup{pdfauthor={Matheus Jonatha},
            pdftitle={Divisibilidade e suas regras},
            pdfsubject={PrepIF - Material de Matemática},
            pdfkeywords={area,planas,geometria,euclidiana,mathjonatha,mthsjonatha,matematica,prepif,if,preparatorio,online,instituto,federal,militares,material},
            pdfproducer={Produzido e gerado no Overleaf},
            pdfcreator={pdflatex}}
\begin{document}
    \import{estrutura/}{wallpaper.tex} % CAPA
        \begin{center}
            {\LARGE {\sc divisibilidade e suas regras}}
        \end{center}

\begin{definicao}
Um \textbf{critério de divisibilidade} é uma regra que permite avaliarmos se um dado número natural é ou não \textbf{divisível} por outro número natural, sem que seja necessário efetuarmos a divisão.
\end{definicao}

\section*{Divisibilidade por 2}
Um número é divisível por 2 quando for par, ou seja,
quando o algarismo das unidades for igual a 0,2,4,6 ou 8.
    \begin{obs}
        Para identificar um número par, basta observarmos o
        algarismo da unidade desse número: números pares têm
        algarismo da unidade igual a 0, 2, 4, 6 ou 8.
    \end{obs}

\begin{exemplo} ~\\
\begin{tasks}(1)
    \task Os números 2742, 234572, 111348, 230 são divisíveis por 2, pois são números pares;
    \task Os números 1777, 2015, 456789, 41253 e 111 não são divisíveis por 2, pois são ímpares.
\end{tasks}
    
\end{exemplo}

\section*{Divisibilidade por 3}
Um número é divisível por 3 quando a soma de seus algarismos for divisível por 3. 

\begin{exemplo}~\\

\begin{tasks}(1)
        \task 111111 é divisível por 3 pois a soma de seus algarismos 1+1+1+1+1+1 = 6 e seis é divisível por 3;
        \task 432 é divisível por 3 pois a soma de seus algarismos 4 + 3 + 2 = 9 é divisível por 3;
        \task 1621 não é divisível por 3 pois a soma de seus algarismos 1+ 6 + 2 +1 = 10 e 10 não é divisível por 3.
    \end{tasks}
\end{exemplo}

\section*{Divisibilidade por 4}
Um número é divisível por 4 quando:
\begin{tasks}(1)
    \task Os dois últimos algarismos que o compõe for divisível por 4. Ou
    \task Os dois últimos algarismos que o compõe forem iguais a zero. 
\end{tasks}


\begin{exemplo}~\\
   \begin{tasks}(1)
        \task 316 é divisível por 4 pois seus dois últimos algarismos, que é 16, é divisível por 4;
        \task 1000 é divisível por 4 pois os seus dois últimos algarismos são iguais a zero;
        \task 215 não é divisível por 4, pois os dois últimos algarismos, 15, não é divisível por 4. 
   \end{tasks}
\end{exemplo}

\newpage
\import{estrutura/}{wallpaper.tex} % CAPA
\section*{Divisibilidade por 5}
Um número é divisível por 5 quando termina em 0 ou
5.
\begin{exemplo}~\\
   \begin{tasks}(1)
        \task 115 é divisível por 5 ,pois termina em 5;
        \task 230 é divisível por 5, pois termina em 0;
        \task 211 não é divisível por 5, pois não termina em 0 e
nem em 5.
   \end{tasks}
\end{exemplo}

\section*{Divisibilidade por 6}
Um número é divisível por 6 quando for divisível por 2
e por 3 ao mesmo tempo. 
\begin{exemplo}~\\
   \begin{tasks}(1)    
    \task 702 é divisível por 6, pois 702 é par , logo divisível
por 2 e a soma de seu algarismos 7 + 0 + 2 = 9 , logo divisível por 3;
    \task 104 é divisível por 2, porém 104 não é divisível por 3, logo 104 não é divisível por 6. 
   \end{tasks}
\end{exemplo}

\section*{Divisibilidade por 7}
Duplicar o algarismo das unidades e subtrair do resto do número. Se o resultado for divisível por 7, o número é divisível por 7. 
\begin{exemplo}~\\
   \begin{tasks}(1)  
        \task 203 : 7 = 29, pois 2*3 = 6 e 20 – 6 = 14
        \task 294 : 7 = 42, pois 2*4 = 8 e 29 – 8 = 21
        \task 840 : 7 = 120, pois 2*0 = 0 e 84 – 0 = 84 
   \end{tasks}
\end{exemplo}

\section*{Divisibilidade por 8}
Todo número será divisível por 8 quando terminar em 000, ou os últimos três números forem divisíveis por 8. 
\begin{exemplo}~\\
   \begin{tasks}(1)        
    \task 1000 : 8 = 125, pois termina em 000
    \task 1208 : 8 = 151, pois os três últimos são divisíveis por 8
   \end{tasks}
\end{exemplo}

\newpage
\import{estrutura/}{wallpaper.tex} % CAPA
\section*{Divisibilidade por 9}
Um número é divisível por 9 quando a soma de seus algarismos for divisível por 9. 
\begin{exemplo}~\\
   \begin{tasks}(1)       
    \task 90 : 9 = 10, pois 9 + 0 = 9
    \task 1125 : 9 = 125, pois 1 + 1 + 2 + 5 = 9
    \task 4788 : 9 = 532, pois 4 + 7 + 8 + 8 = 27
   \end{tasks}
\end{exemplo}

\section*{Divisibilidade por 10, 100, 1000,...}
Um número é divisível por 10 quando termina em zero, em 100 quando termina em dois zeros, em 1000 quando
termina em três zeros, etc. 
\begin{exemplo}~\\
   \begin{tasks}(1)
       \task 30 é divisível por 10, pois termina em zero;
        \task 200 é divisível por 100, pois termina com dois zeros;
        \task 432000 é divisível por 1000, pois termina em três
zeros. 
   \end{tasks}
\end{exemplo}




\vfill
\section*{Referências}
\noindent Disponível em: \url{https://portaldaobmep.impa.br/uploads/material_teorico/5obdvodxmnk8c.pdf}. Acesso em: 08 de setembro de 2020.

\noindent Disponível em: 
\url{https://repositorio.ufsc.br/xmlui/bitstream/handle/123456789/97082/Luis_Junqueira.PDF}. Acesso em: 09 de setembro de 2020.

\noindent Disponível em: 
\url{https://profmariocastro.files.wordpress.com/2015/02/05-divisibilidade.pdf}. Acesso em: 09 de setembro de 2020.

\noindent Disponível em: 
\url{http://portaldoprofessor.mec.gov.br/storage/materiais/0000016819.PDF}. Acesso em: 10 de setembro de 2020.


\end{document}
