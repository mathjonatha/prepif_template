\documentclass[fleqn,10pt]{article}
\usepackage[T1]{fontenc}
\usepackage[utf8]{inputenc}
\usepackage[brazil]{babel}
%\usepackage[math]{anttor} % fonte um pouco mais estilizada
\usepackage{import}
%\usepackage{parskip}
%=========================Packages==================================%
\usepackage{lscape,booktabs,latexsym,multicol,lmodern, natbib,graphicx,tikz,tkz-euclide,enumitem,fancyhdr, lipsum,siunitx, setspace,float}

\usepackage{amsmath,amsfonts,amssymb,amsthm}
\everymath{\displaystyle}

% configurações das questões, bem como: pontuação e estrutura.

\usepackage{tasks} % cria lista curta
\usepackage{exsheets} % cria questoes
\SetupExSheets[points]{name=ponto/s,number-format=\color{blue}} % define as configurações de pontuação das questões, e a cor da pontuação.

\DeclareInstance{exsheets-heading}{fancy-wp}{default}{
toc-reversed = true ,
indent-first = true ,
vscale = 2 ,
pre-code = \rule{\linewidth}{1pt} ,
post-code = \rule{\linewidth}{1pt} ,
title-format = \large\scshape\color{rgb:red,0.65;green,0.04;blue,0.07} ,
number-format = \large\bfseries\color{rgb:red,0.02;green,0.04;blue,0.48} ,
points-format = \itshape ,
points-pre-code = ( ,
points-post-code = ) ,
join =
{
number[r,B]title[l,B](.333em,0pt) ;
number[r,B]points[l,B](.333em,0pt)
} ,
attach = { main[hc,vc]number[hc,vc](0pt,0pt) }
}

%\SetupExSheets{headings=fancy-wp} % estilo diferente para o topo do enunciado com o nome " Exercício




\graphicspath{{imgs/}} %informa a pasta em que as imagens estão
%\usepackage{showframe} %Mostra linhas de marcação e margens
%FONTES
%\usepackage{fontspec}
%\usepackage{avant}
%\usepackage{mathptmx}
\usepackage{times}
%\usepackage[classicRfeIm]{kpfonts}
%\usepackage{kurier}


\usepackage{hyperref}% add hypertext capabilities
%\usepackage{txfonts}
%\usepackage{mathrsfs}

% CORES
\usepackage{xcolor}
\definecolor{corprimaria}{RGB}{10,48,123} 
\definecolor{corsecundaria}{RGB}{116,23,255}
\definecolor{corlinha}{RGB}{47,158,65}

\definecolor{corexercicio}{RGB}{38,38,38}
\definecolor{cordefinicao}{RGB}{47,158,65}
% \definecolor{corexemplo}{RGB}{116,23,255}
\definecolor{corexemplo}{RGB}{38,38,38}

% CORES DO PREPIF
\definecolor{cor1}{RGB}{239,239,239}
\definecolor{cor2}{RGB}{205,25,30}
\definecolor{cor3}{RGB}{47,158,65}
\definecolor{cor4}{RGB}{38,38,38}
\definecolor{cor5}{RGB}{28,28,28}

% ESTILOS
\newtheoremstyle{geral}% Nome do estilo do teorema
{0pt}% Espaço acima
{0pt}% Espaço abaixo
{\normalfont}% Fonte do corpo
{}% Indent amount
{\small\bf\sffamily\color{cor4}}% Theorem head font
{:}% Punctuation after theorem head
{0.25em}% Space after theorem head
{} % Optional theorem note

\newcounter{dummy}
\newcounter{Exer}
\newcounter{exer}
\newcounter{exem}
\theoremstyle{geral}
\newtheorem{ExercicioT}[Exer]{Exercício resolvido}
% \newtheorem{exercicioT}[exer]{ }
\newtheorem{definicaoT}[dummy]{Conceito}
\newtheorem{exemploT}[exem]{Exemplo}
\newtheorem{obsT}{Observação}

\RequirePackage[framemethod=default]{mdframed} % Required for creating the theorem, definition, exercise and corollary boxes

% Caixa de exercícios
\newmdenv[skipabove=7pt,
skipbelow=7pt,
rightline=false,
leftline=true,
topline=false,
bottomline=false,
backgroundcolor=corexercicio!10,
linecolor=corexercicio,
innerleftmargin=5pt,
innerrightmargin=5pt,
innertopmargin=5pt,
innerbottommargin=5pt,
leftmargin=0cm,
rightmargin=0cm,
linewidth=4pt]{caixaEx}	

% Caixa de definição
\newmdenv[skipabove=7pt,
skipbelow=7pt,
rightline=false,
leftline=true,
topline=false,
bottomline=false,
backgroundcolor=cordefinicao!10,
linecolor=cordefinicao,
innerleftmargin=5pt,
innerrightmargin=5pt,
innertopmargin=5pt,
innerbottommargin=5pt,
leftmargin=0cm,
rightmargin=0cm,
linewidth=4pt]{caixaDE}	

% Caixa de exemplos
\newmdenv[skipabove=7pt,
skipbelow=7pt,
rightline=false,
leftline=true,
topline=false,
bottomline=false,
backgroundcolor=corexemplo!10,
linecolor=corexemplo,
innerleftmargin=5pt,
innerrightmargin=5pt,
innertopmargin=5pt,
innerbottommargin=5pt,
leftmargin=0cm,
rightmargin=0cm,
linewidth=4pt]{caixaExem}	

% Caixa de observações	  
\newmdenv[skipabove=7pt,
skipbelow=7pt,
rightline=false,
leftline=true,
topline=false,
bottomline=false,
backgroundcolor=cor2!10,
linecolor=cor2,
innerleftmargin=5pt,
innerrightmargin=5pt,
innertopmargin=5pt,
innerbottommargin=5pt,
leftmargin=0cm,
rightmargin=0cm,
linewidth=4pt]{caixaObs}


\newenvironment{Exercicio}{\begin{caixaEx}\begin{ExercicioT}}{\end{ExercicioT}\end{caixaEx}}
% \newenvironment{exercicio}{\begin{caixaEx}\begin{exercicioT}}{\end{exercicioT}\end{caixaEx}}
\newenvironment{definicao}{\begin{caixaDE}\begin{definicaoT}}{\end{definicaoT}\end{caixaDE}}
\newenvironment{exemplo}{\begin{caixaExem} \begin{exemploT}}{\end{exemploT}\end{caixaExem}}

\newenvironment{obs}{\begin{caixaObs}\begin{obsT}}{\end{obsT}\end{caixaObs}}

\newenvironment{exercicio}[2][{\color{corexercicio}Exercício}]{\begin{trivlist}

\item[\hskip \labelsep {\bfseries #1}\hskip \labelsep {\bfseries #2.}]}{\end{trivlist}}


%===================================================
% MARGINS
%\usepackage[top=8mm, bottom=20mm, left=8mm, right=8mm]{geometry}
\usepackage{geometry}
\geometry{
	paper=a4paper, 
	top=2.5cm, 
	bottom=2.5cm, 
	left=1.5cm, 
	right=2cm,
	headheight=14pt, % Header height
	footskip=1.4cm, % Espaço da margem inferior à linha de base do rodapé
	headsep=10pt, % Espaço da margem superior até a linha de base do cabeçalho
	%showframe, % Uncomment to show how the type block is set on the page
}


% NOVOS COMANDOS
\newcommand{\atv}{Lista de Exercícios 00}
\newcommand{\preceptor}{Monitor: Matheus Jonatha}




% CABEÇALHO E RODAPÉ
\pagestyle{fancy}
%\lfoot{\notaesquerda}
\cfoot{}
\rfoot{{\color{white}\thepage}}
%\lhead{HELLO}
%\chead{HELLO}
%\rhead{\textbf{The performance of new graduates}}
\renewcommand{\headrulewidth}{0pt} %linha horizontal no topo da pagina
%\renewcommand{\footrulewidth}{0.4pt} %linha horizontal no pé da pagina

%\setlength\parindent{0pt}
%\setlength\parskip{1.5ex}
%\setlength\parsep{1.5\parskip}
%\thispagestyle{empty}%\bigskip %Rodapé na primeira pagina


%para nao ficar o retangulo em volta dos links, apenas muda a cor dos caracteres
\hypersetup{ colorlinks,
linkcolor=blue,
filecolor=blue,
urlcolor=blue,
citecolor=blue }



% QUESTÕES
% configurações das questões, bem como: pontuação e estrutura.

\usepackage{tasks} % cria lista curta
\usepackage{exsheets} % cria questoes
\SetupExSheets[points]{ name=ponto/s,number-format=\color{blue}} % define as configurações de pontuação das questões, e a cor da pontuação.

\DeclareInstance{exsheets-heading}{fancy-wp}{default}{
toc-reversed = true ,
indent-first = true ,
vscale = 2 ,
pre-code = \rule{\linewidth}{1pt} ,
post-code = \rule{\linewidth}{1pt} ,
title-format = \large\scshape\color{rgb:red,0.65;green,0.04;blue,0.07} ,
number-format = \large\bfseries\color{rgb:red,0.02;green,0.04;blue,0.48} ,
points-format = \itshape ,
points-pre-code = ( ,
points-post-code = ) ,
join =
{
number[r,B]title[l,B](.333em,0pt) ;
number[r,B]points[l,B](.333em,0pt)
} ,
attach = { main[hc,vc]number[hc,vc](0pt,0pt) }
}

%\SetupExSheets{headings=fancy-wp} % estilo diferente para o topo do enunciado com o nome " Exercício
\usepackage{capt-of}%%To get the caption

%\hypersetup{pdftitle={Produtos Notáveis},pdfauthor={Matheus Jonatha}}
\hypersetup{pdfauthor={Matheus Jonatha},
            pdftitle={Produtos Notáveis},
            pdfsubject={PrepIF - Material de Matemática},
            pdfkeywords={produtos,notaveis,mathjonatha,mthsjonatha,matematica,prepif,if,preparatorio,online,instituto,federal,militares,material},
            pdfproducer={Produzido e gerado no Overleaf},
            pdfcreator={pdflatex}}

\begin{document}
    \import{estrutura/}{wallpaper.tex} % CAPA
        \begin{center}
            {\LARGE {\sc produtos notáveis}}
        \end{center}

%\section*{Introdução}

\begin{definicao}
    Os produtos notáveis são expressões equivalentes a produtos populares (por isso o nome ``notáveis''). Elas são utilizadas com o objetivo de tornar os cálculos mais simples e mais rápidos.
\end{definicao}
\section*{Os produtos notáveis mais utilizados}
 Como são muito frequentes no cálculo, iremos listar os que costumam ser mais utilizados em provas dos institutos federais:

    \begin{enumerate}[label=\textbf{(\Roman*)}]
        \item Quadrado da soma de dois termos
            \[ (a+b)^2 = a^2 + 2ab + b^2\]
            Importante lembrar que podemos escrever \( (a+b)^2\) como sendo \( (a+b) \times (a+b) \).
        \item Quadrado da diferença de dois termos
            \[ (a-b)^2 = a^2 - 2ab + b^2\]
            Lembre-se, ocorre o mesmo que o produto notável anterior, podemos escrever \( (a-b)^2\) como sendo \( (a-b) \times (a-b) \).
        \item Produto da soma pela diferença de dois termos
            \[ (a+b) \times (a-b)= a^2-b^2 \]
        
        \begin{obs}
        Os produtos notáveis a seguir não são tão cobrados quanto os anteriores, mas é interessante conhecê-los.
        \end{obs}
        \item Cubo da soma de dois termos
            \[ (a+b)^3 \textrm{ ou } (a+b) \times (a+b) \times (a+b) = a^3 + 3a^{2}b + 3ab^2 + b^3 \]
        \item Cubo da diferença de dois termos
            \[ (a-b)^3 \textrm{ ou } (a-b) \times (a-b) \times (a-b) = a^3-3a^{2}b + 3ab^2 - b^3 \]
    \end{enumerate}
    

% \begin{exercicio}{\color{corexercicio}1}
% \text{{\bf IFRN - 2015}}\\
%     Questão sobre funções
% \end{exercicio}

\begin{Exercicio}
Desenvolver os produtos notáveis e simplificar a expressão:
\begin{tasks}(1)
        \task \( \left( \frac{a}{3}-b \right)^2 \)
        
        {\bf Resolução:}
            \[ \left( \frac{a}{3}-b \right)^2 = \left( \frac{a}{3} \right)^2 - 2 \cdot \frac{a}{3} \cdot b + b^2 = \frac{a^2}{9} - \frac{2ab}{3} + b^2 \]
        Note que utilizamos o \textbf{quadrado da diferença de dois termos} e ao final simplificamos a expressão. Um ponto importante é que ao final das questões sempre simplifique, pois a alternativa no gabarito estará escrita na forma simplificada.
        \task \( ( x + y )^2 - y^2 \)
        
        {\bf Resolução:}
            \[ ( x + y )^2 - y^2 = x^2+2xy+y^2 -y^2 = x^2+2xy = x \cdot (x+2y) \]
        Na resolução utilizamos o \textbf{quadrado da soma de dois termos}. Note que no final usamos \textbf{fatoração} para simplificar a expressão, e esse é um tema importante para te auxiliar no reconhecimento de produtos notáveis e simplificar. 
    \end{tasks}
\end{Exercicio}

\newpage
\import{estrutura/}{wallpaper.tex} % CAPA

 \begin{center}
            {\LARGE {\sc hora de praticar}}
        \end{center}

Nesta seção, você encontra exercícios de fixação. Tente realizar todos, pois o conteúdo mostrado anteriormente é utilizado como ferramenta no desenvolvimento das questões. É importante lembrar que uma questão dificilmente vai te pedir para desenvolver um produto notável, já que esse conteúdo é utilizado como \textbf{ferramenta} para encontrar o resultado final (não é o único caminho, mas é um atalho).

As resoluções das questões podem ser encontradas no nosso \href{https://prepif.herokuapp.com/instituicoes}{site}. No enunciado das questões, são informados o instituto, o ano e o número da questão (ex.: \textbf{IFRN2020Q09}).

\begin{multicols}{2}
\setlength\columnseprule{1pt}
\def\columnseprulecolor{\color{corlinha}}%

\begin{question}[name = Questão]
\textbf{IFRS2015Q06}
Um retângulo tem dimensões \( x \) e \( y \), que são expressas pelas equações \( x^2=12\) e \( (y-1)^2=3\). O perímetro e a área deste retângulo são, respectivamente

\begin{tasks}(1)
        \task \( 6\sqrt{3} + 2 \) e \( 2 + 6\sqrt{3} \)
        \task \( 6\sqrt{3}\) e \( 1 + 2\sqrt{3} \)
        \task \( 6\sqrt{3} +2 \) e \( 12 \)
        \task \( 6 \) e \( 2\sqrt{3} \)
        \task \( 6\sqrt{3} + 2 \) e \( 2\sqrt{3} + 6 \)
    \end{tasks}
\end{question}

\begin{question}[name = Questão]
\textbf{IFCE2018Q32}
Para os números \( x=\frac{2}{5} \), \( y= \frac{3}{7} \) e \( z= \frac{1}{3} \), quando escrevemos \( \left( \frac{x}{y} - z \right) \) como fração irredutível, obtemos numerador e denominador que somam.

\begin{tasks}(2)
        \task \( 24 \)
        \task \( 12 \)
        \task \( 15 \)
        \task \( 34 \)
        \task \( 52 \)
    \end{tasks}
\end{question}

\begin{question}[name = Questão]
\textbf{IFCE2017Q32}
O número natural \( N \) deixa resto \( 3 \) na divisão por \( 5 \). O resto da divisão de \(8\cdot N^2\) por \( 5 \) é igual a

\begin{tasks}(1)
        \task \( 1 \)
        \task \( 3 \)
        \task \( 4 \)
        \task \( 2 \)
        \task \( 0 \)
    \end{tasks}
\end{question}

\begin{question}[name = Questão]
\textbf{IFCE2017Q37}
Para todo número real \( x \), com \( x \neq 1\) e \( x \neq -1\), a expressão \( \frac{(x^3-1)(x+1)}{x^2-1} \) é igual a

\begin{tasks}(1)
        \task \( x-1 \)  
        \task \( x^2-1 \)  
        \task \( x^2+x+1 \)  
        \task \( x^2-x-1 \)  
        \task \( x+1 \)  
    \end{tasks}
\end{question}

\begin{question}[name = Questão]
\textbf{IFCE20172Q03}
O número real \( b \) é tal que \( b^2=3b-1 \).Nessas condições, é verdade que

\begin{tasks}(1)
        \task \( b^4=9b-1 \)  
        \task \( b^4=16b-4 \)  
        \task \( b^4=21b-8 \)  
        \task \( b^4=12b-4 \)  
        \task \( b^4=3b+1 \)  
    
    \end{tasks}
\end{question}

\begin{question}[name = Questão]
\textbf{IFBA2018Q14}
A hipotenusa do triângulo retângulo de catetos \( (4 + \sqrt{5})\mathrm{cm}\) e \( (4 + \sqrt{5})\mathrm{cm} \) é igual ao lado de um quadrado de área \( x ~ \mathrm{cm^2} \). Determine o valor de \( x \).

\begin{tasks}(1)
        \task \( 34 \)  
        \task \( 20 \)  
        \task \( 81 \)  
        \task \( 42 \)  
        \task \( 54 \)  
    
    \end{tasks}
\end{question}

\end{multicols}

\vfill
\section*{Referências}
\noindent Disponível em: \url{https://www.todamateria.com.br/produtos-notaveis/}. Acesso em: 27 de julho de 2020.

\noindent Disponível em: \url{https://bit.ly/3g5G1vz}. Acesso em: 27 de julho de 2020.

\noindent Disponível em: \url{https://prepif.herokuapp.com/instituicoes}. Acesso em: 28 de julho de 2020.



\end{document}